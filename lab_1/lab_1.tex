\documentclass[12pt, a4paper]{report}
\usepackage[utf8x]{inputenc}
\usepackage[T2A]{fontenc}
\usepackage[english,russian]{babel}

\usepackage{graphicx}
\usepackage{listings}
\usepackage{color}

\usepackage{amsmath}
\usepackage{pgfplots}
\usepackage{url}
\usepackage{flowchart}
\usepackage{float}
\usepackage{tikz}
\usepackage{multirow}
\usepackage{graphicx}
\usepackage[
bookmarks=true, colorlinks=true, unicode=true,
urlcolor=black,linkcolor=black, anchorcolor=black,
citecolor=black, menucolor=black, filecolor=black,
]{hyperref}

\DeclareGraphicsExtensions{.pdf,.png,.jpg,.svg}
\usetikzlibrary{shapes, arrows}

\usepackage{pgfplotstable}

\renewcommand\contentsname{Содержание}

\usepackage{geometry}
\geometry{top=15mm}
\geometry{right=15mm}
\geometry{left=15mm}
\geometry{bottom=20mm}
\geometry{ignorefoot}


\lstset{ %
	language=Lisp,                 % выбор языка для подсветки (здесь это С)
	basicstyle=\small\sffamily, % размер и начертание шрифта для подсветки кода
	numbers=left,               % где поставить нумерацию строк (слева\справа)
	numberstyle=\tiny,           % размер шрифта для номеров строк
	stepnumber=1,                   % размер шага между двумя номерами строк
	numbersep=-5pt,                % как далеко отстоят номера строк от         подсвечиваемого кода
	backgroundcolor=\color{white}, % цвет фона подсветки - используем         \usepackage{color}
	showspaces=false,            % показывать или нет пробелы специальными     отступами
	showstringspaces=false,      % показывать или нет пробелы в строках
	showtabs=false,             % показывать или нет табуляцию в строках
	frame=single,              % рисовать рамку вокруг кода
	tabsize=2,                 % размер табуляции по умолчанию равен 2 пробелам
	captionpos=t,              % позиция заголовка вверху [t] или внизу [b] 
	breaklines=true,           % автоматически переносить строки (да\нет)
	breakatwhitespace=false, % переносить строки только если есть пробел
	escapeinside={\%*}{*)},   % если нужно добавить комментарии в коде
	keywordstyle=\color{blue}\ttfamily,
	stringstyle=\color{red}\ttfamily,
	commentstyle=\color{green}\ttfamily,
	morecomment=[l][\color{magenta}]{\#},
	columns=fullflexible }

\usepackage{titlesec, blindtext, color}
\setcounter{secnumdepth}{-1}
\titleformat{\chapter}[hang]{\LARGE\bfseries}{}{0pt}{\LARGE\bfseries}
\titleformat{\section}{\bfseries\Large}{}{-1cm}{\centering}
\titleformat{\subsection}{\bfseries\normalsize}{}{-1cm}{\centering}



\begin{document}
	
	\begin{titlepage}
		
		\begin{table}[H]
			\centering
			\footnotesize
			\begin{tabular}{cc}
				\multirow{8}{*}{\includegraphics[scale=0.35]{img/bmstu.jpg}}
				& \\
				& \\
				& \textbf{Министерство науки и высшего образования Российской Федерации} \\
				& \textbf{Федеральное государственное бюджетное образовательное учреждение} \\
				& \textbf{высшего образования} \\
				& \textbf{<<Московский государственный технический} \\
				& \textbf{университет имени Н.Э. Баумана>>} \\
				& \textbf{(МГТУ им. Н.Э. Баумана)} \\
				& \textbf{} \\
			\end{tabular}
		\end{table}
		
		\vspace{-2.5cm}
		
		\begin{flushleft}
			\rule[-1cm]{\textwidth}{3pt}
			\rule{\textwidth}{1pt}
		\end{flushleft}
		
		\begin{flushleft}
			\small
			ФАКУЛЬТЕТ
			\underline{<<Информатика и системы управления>>\ \ \ \ \ \ \ 
				\ \ \ \ \ \ \ \ \ \ \ \ \ \ \ \ \ \ \ \ \ \ \ \ \ \ \ \ \ \ \ 
				\ \ \ \ \ \ \ \ \ \ \ \ \ \ \ } \\
			КАФЕДРА
			\underline{<<Программное обеспечение ЭВМ и
				информационные технологии>>
				\ \ \ \ \ \ \ \ \ \ \ \ \ \ \ \ \ \ \ \ }
		\end{flushleft}
		
		\vspace{2cm}
		
		\begin{center}
			\textbf{Лабораторная работа № 1} \\
			\vspace{0.5cm}
			\textbf{Вариант 22}
		\end{center}
		
		\vspace{4cm}
		
		\begin{flushleft}
			\begin{tabular}{ll}
				\textbf{Дисциплина} & Математическая статистика. \\
				\textbf{Тема} & \\
				\textbf{Студент} & Тимонин А. С. \\
				\textbf{Группа} & ИУ7-62Б \\
				\textbf{Оценка (баллы)} & \\
				\textbf{Преподаватель} & Власов П.А. \\
			\end{tabular}
		\end{flushleft}
		
		\vspace{6cm}
		
		\begin{center}
			Москва, 2020 г.
		\end{center}
		
		
	\end{titlepage}
	
	
	\section{Формулы для вычисления}
	
	\hspace{0.7cm}\textbf{Для генеральной совокупности $\vec{x} = {(x_1, \dots, x_n)}$}
	
	\vspace{0.5cm}\textbf{Формула для вычисления максимального значения $M_{\max}$:}
	
	\begin{equation*} \label{Mmax}
	M_{\max} = \max{(x_1, \dots, x_n)}
	\end{equation*}
	
	\textbf{Формула для вычисления минимального значения $M_{\min}$:}
	
	\begin{equation*} \label{Mmin}
	M_{\min} = \min{(x_1, \dots, x_n)}
	\end{equation*}
	
	\textbf{Размах выборки R  считается по формуле:}
	
	\begin{equation*} \label{R}
	R = M_{\max} - M_{\min}
	\end{equation*}
	
	\textbf{Вычисление оценки математического ожидания MX:}
	
	\begin{equation*}
	\hat{\mu} = \vec{x} = \frac{1}{n}\sum_{i=1}^{n} x_i
	\end{equation*}
	
	\textbf{Вычисление оценки дисперсии DX:}
	
	\begin{equation*}
	S^2 = \frac{1}{n-1} \sum_{i=1}^n (x_i - \overline{x})^2
	\end{equation*}
	
	\section{Определение имперической плотности и гистограммы}
	
	\hspace{0.5cm} \textbf{Интервальный статистический ряд}
	
	Пусть $\vec{x}$ - выборка из генеральной совокупности X. Если объем n этой выборки велик ($n\geq50$), то значения $x_i$ группируют не только в статистический ряд, но и в так называемый \underline{интервальный статистический ряд}. Для этого отрезок $J = [x_{(1)}, x_{(n))}]$ делят на p равновеликих частей:
	
	\begin{equation*}
	J_i = [a_i,a_{i+1}), i = \overline{0;p-i}
	\end{equation*}
	
	\begin{equation*}
	J_p = [a_{p-1},a_{p}]
	\end{equation*}
	
	\vspace{0.5cm}где $a_i = x_{(1)} + i\Delta,\; t = \overline{0;p}, \Delta = \frac{|J|}{p} = \frac{x_{(n)} - x_{(1)}}{p}$
	
	\vspace{0.5cm}\textbf{\underline{Опр} Интервальным статистическим рядом} называют таблицу
	
	\begin{table}[H]
		\centering
		\begin{tabular}{|c|c|c|c|c|}
			\hline
			$J_1$ & $\dots$ & $J_i$ & $\dots$ & $J_p$ \\
			\hline
			$n_1$ & $\dots$ & $n_i$ & $\dots$ & $n_p$ \\
			\hline
		\end{tabular}
	\end{table}
	
	Здесь $n_i$- количество элементов выборки $\vec{x}$, которые $\in J_i$
	
	\vspace{0.3cm}\textbf{\underline{Замечание}}
	
	\begin{enumerate}
		\item Очевидно, что $\sum_{i=1}^p n_i = n$
		\item Для выборки p - числа интервалов можно пользоваться формулой $p = [\log_n n]+1$
	\end{enumerate}
	
	где [a] - целая часть числа a
	
	\vspace{0.5cm}\textbf{\underline{Опр} Эмпирической плотностью} (отвечающей выборке $\vec{x}$) называют функцию:
	
	\begin{equation*}
	\hat f_n(x) =
	\begin{cases}
	\frac{n_i}{n \Delta}, x \in J_i, i = \overline{1; p} \\
	0, \text{ иначе} \\
	\end{cases}
	\end{equation*}
	
	\textbf{\underline{Опр} Гистограммой} называют график эмпирической плотности
	
	
	\section{Определение имперической плотности и гистограммы}
	
	
\end{document}