\documentclass[12pt, a4paper]{report}
\usepackage[utf8x]{inputenc}
\usepackage[T2A]{fontenc}
\usepackage[english,russian]{babel}

\usepackage{graphicx}
\usepackage{listings}
\usepackage{color}

\usepackage{amsmath}
\usepackage{amssymb}
\usepackage{pgfplots}
\usepackage{url}
\usepackage{flowchart}
\usepackage{float}
\usepackage{tikz}
\usepackage{multirow}
\usepackage{graphicx}
\usepackage[
bookmarks=true, colorlinks=true, unicode=true,
urlcolor=black,linkcolor=black, anchorcolor=black,
citecolor=black, menucolor=black, filecolor=black,
]{hyperref}

\DeclareGraphicsExtensions{.pdf,.png,.jpg,.svg}
\usetikzlibrary{shapes, arrows}

\usepackage{pgfplotstable}

\renewcommand\contentsname{Содержание}

\usepackage{geometry}
\geometry{top=15mm}
\geometry{right=15mm}
\geometry{left=15mm}
\geometry{bottom=20mm}
\geometry{ignorefoot}


\lstset{ %
	language=Matlab,                 % выбор языка для подсветки (здесь это С)
	basicstyle=\small\sffamily, % размер и начертание шрифта для подсветки кода
	numbers=left,               % где поставить нумерацию строк (слева\справа)
	numberstyle=\tiny,           % размер шрифта для номеров строк
	stepnumber=1,                   % размер шага между двумя номерами строк
	numbersep=-5pt,                % как далеко отстоят номера строк от         подсвечиваемого кода
	backgroundcolor=\color{white}, % цвет фона подсветки - используем         \usepackage{color}
	showspaces=false,            % показывать или нет пробелы специальными     отступами
	showstringspaces=false,      % показывать или нет пробелы в строках
	showtabs=false,             % показывать или нет табуляцию в строках
	frame=single,              % рисовать рамку вокруг кода
	tabsize=2,                 % размер табуляции по умолчанию равен 2 пробелам
	captionpos=t,              % позиция заголовка вверху [t] или внизу [b] 
	breaklines=true,           % автоматически переносить строки (да\нет)
	breakatwhitespace=false, % переносить строки только если есть пробел
	escapeinside={\%*}{*)},   % если нужно добавить комментарии в коде
	keywordstyle=\color{blue}\ttfamily,
	stringstyle=\color{red}\ttfamily,
	commentstyle=\color{green}\ttfamily,
	morecomment=[l][\color{magenta}]{\#},
	columns=fullflexible }

\usepackage{titlesec, blindtext, color}
\setcounter{secnumdepth}{-1}
\titleformat{\chapter}[hang]{\LARGE\bfseries}{}{0pt}{\LARGE\bfseries}
\titleformat{\section}{\bfseries\Large}{}{-1cm}{\centering}
\titleformat{\subsection}{\bfseries\normalsize}{}{-1cm}{\centering}



\begin{document}
	
	\begin{titlepage}
		
		\begin{table}[H]
			\centering
			\footnotesize
			\begin{tabular}{cc}
				\multirow{8}{*}{\includegraphics[scale=0.35]{img/bmstu.jpg}}
				& \\
				& \\
				& \textbf{Министерство науки и высшего образования Российской Федерации} \\
				& \textbf{Федеральное государственное бюджетное образовательное учреждение} \\
				& \textbf{высшего образования} \\
				& \textbf{<<Московский государственный технический} \\
				& \textbf{университет имени Н.Э. Баумана>>} \\
				& \textbf{(МГТУ им. Н.Э. Баумана)} \\
				& \textbf{} \\
			\end{tabular}
		\end{table}
		
		\vspace{-2.5cm}
		
		\begin{flushleft}
			\rule[-1cm]{\textwidth}{3pt}
			\rule{\textwidth}{1pt}
		\end{flushleft}
		
		\begin{flushleft}
			\small
			ФАКУЛЬТЕТ
			\underline{<<Информатика и системы управления>>\ \ \ \ \ \ \ 
				\ \ \ \ \ \ \ \ \ \ \ \ \ \ \ \ \ \ \ \ \ \ \ \ \ \ \ \ \ \ \ 
				\ \ \ \ \ \ \ \ \ \ \ \ \ \ \ } \\
			КАФЕДРА
			\underline{<<Программное обеспечение ЭВМ и
				информационные технологии>>
				\ \ \ \ \ \ \ \ \ \ \ \ \ \ \ \ \ \ \ \ }
		\end{flushleft}
		
		\vspace{2cm}
		
		\begin{center}
			\textbf{Лабораторная работа № 1} \\
			\vspace{0.5cm}
			\textbf{Вариант 22}
		\end{center}
		
		\vspace{4cm}
		
		\begin{flushleft}
			\begin{tabular}{ll}
				\textbf{Дисциплина} & Математическая статистика. \\
				\textbf{Тема} & \\
				\textbf{Студент} & Тимонин А. С. \\
				\textbf{Группа} & ИУ7-62Б \\
				\textbf{Оценка (баллы)} & \\
				\textbf{Преподаватель} & Власов П.А. \\
			\end{tabular}
		\end{flushleft}
		
		\vspace{6cm}
		
		\begin{center}
			Москва, 2020 г.
		\end{center}
		
		
	\end{titlepage}
	
	
	\section{Теоретическая часть}
	\subsection{Формулы для вычисления}
	
	\hspace{0.7cm} $\vec{x} = {(x_1, \dots, x_n)}$ \textbf{- выборка из генеральной совокупности} $\vec{X}$
	
	\vspace{0.5cm}\textbf{Формула для вычисления максимального значения $M_{\max}$:}
	
	\begin{equation*} \label{Mmax}
	M_{\max} = \max{(x_1, \dots, x_n)}
	\end{equation*}
	
	\textbf{Формула для вычисления минимального значения $M_{\min}$:}
	
	\begin{equation*} \label{Mmin}
	M_{\min} = \min{(x_1, \dots, x_n)}
	\end{equation*}
	
	\textbf{Размах выборки R вычисляется по формуле:}
	
	\begin{equation*} \label{R}
	R = M_{\max} - M_{\min}
	\end{equation*}
	
	\textbf{Вычисление оценки математического ожидания MX:}
	
	\begin{equation*}
	\hat{\mu}(\hat{X_n}) = \overline{X}_n = \frac{1}{n}\sum_{i=1}^{n} x_i
	\end{equation*}
	
	\textbf{Вычисление оценки дисперсии DX:}
	
	\begin{equation*}
	S^2 = \frac{1}{n-1} \sum_{i=1}^n (x_i - \overline{x})^2
	\end{equation*}
	
	\vspace{0.5cm}
	\subsection{Определение эмперической плотности и гистограммы}
	
	\hspace{0.5cm} \textbf{Интервальный статистический ряд}
	
	Пусть $\vec{x}$ - выборка из генеральной совокупности X. Если объем n этой выборки велик ($n\geq50$) отрезок $[x_{(1)}; x_{(n)}]$ разбивают на m равновеликих интервалов и для каждого интервала указывают количество принадлежащих элементов выборки. При этом, как правило, m определяют как:  $m = [\log_2n] + 1$ - количество интервалов, $\Delta = \frac{x_{(n)} - x_{(1)}}{m}$ - ширина одного интервала.
	
	Интервалы: 
	
	\begin{equation*}
	J_i = [   x_{(1)}  + (i-1)\Delta; x_{(1)} + i\Delta  ), i = \overline{1;m-1}
	\end{equation*}
	
	\begin{equation*}
	J_m = [   x_{(1)}  + (m-i)\Delta; x_{(1)} + m\Delta    ]
	\end{equation*}
	
	\vspace{0.5cm}где $x_{(n)} = x_{(1)} + m\Delta$
	
	\vspace{0.5cm}\textbf{\underline{Опр} Интервальным статистическим рядом} называют таблицу
	
	\begin{table}[H]
		\centering
		\begin{tabular}{|c|c|c|c|c|}
			\hline
			$J_1$ & $\dots$ & $J_i$ & $\dots$ & $J_p$ \\
			\hline
			$n_1$ & $\dots$ & $n_i$ & $\dots$ & $n_p$ \\
			\hline
		\end{tabular}
	\end{table}
	
	Здесь $n_i$- количество элементов выборки $\vec{x}$, которые $\in J_i$
	
	\vspace{0.3cm}\textbf{\underline{Замечание}}
	
	\begin{enumerate}
		\item Очевидно, что $\sum_{i=1}^p n_i = n$
		\item Для выборки p - числа интервалов можно пользоваться формулой $p = [\log_n n]+1$
	\end{enumerate}
	
	где [a] - целая часть числа a
	
	\vspace{0.5cm}\textbf{\underline{Опр} Эмпирической плотностью} (отвечающей выборке $\vec{x}$) называют функцию:
	
	\begin{equation*}
	\hat f_n(x) =
	\begin{cases}
	\frac{n_i}{n \Delta}, x \in J_i, i = \overline{1; p} \\
	0, \text{ иначе} \\
	\end{cases}
	\end{equation*}
	
	\textbf{\underline{Опр} Гистограммой} называют график эмпирической плотности
	
	
	\vspace{0.5cm}
	\subsection{Определение эмперической функции распределения}
	
	\hspace{0.5cm}
	
	\textbf{\underline{Опр} Эмпирической функцией распределения} называют функцию
	
	\begin{equation*}
	\mathcal{F}_n: \mathbb{R} \to \mathbb{R}
	\end{equation*}
	
	\hspace{1cm} определенную условием
	
	\begin{equation*}
		\hat F_n(x) = \frac{h(x, \vec{x})}{n}
	\end{equation*}
	
	где $h(x, \vec{x})$  - количество элементов выборки $\vec{x}$, которые имеют значение, меньшее x.
	
	\newpage
	
	\section{Практическая часть}
	
	\subsection{Листинги программы}
	\begin{lstlisting}[caption=Реализация программы]
	function lab1()
			X = [7.76,5.96,4.58,6.13,5.05,6.40,7.46,5.55,5.01,3.79, ...
			7.65,8.87,5.94,7.25,6.76,6.92,6.68,4.89,7.47,6.53, ...
			6.76,6.96,6.58,7.92,8.47,6.27,8.05,5.24,5.60,6.69, ...
			7.55,6.02,7.34,6.81,7.22,6.39,6.40,8.28,5.39,5.68, ...
			6.71,7.89,5.69,5.18,7.84,7.18,7.54,6.04,4.58,6.82, ...
			4.45,6.75,5.28,7.42,6.88,7.10,5.24,9.12,7.37,5.50, ...
			5.52,6.34,5.31,7.71,6.88,6.45,7.51,6.21,7.44,6.15, ...
			6.25,5.59,6.68,6.52,4.03,5.35,6.53,3.68,5.91,6.68, ...
			6.18,7.80,7.17,7.31,4.48,5.69,7.11,6.87,6.14,4.73, ...
			6.60,5.61,7.32,6.75,6.28,6.41,7.31,6.68,7.26,7.94, ...
			7.67,4.72,6.01,5.79,7.38,5.98,5.36,6.43,7.25,5.54, ...
			6.66,6.47,6.84,6.13,6.21,5.52,6.33,7.55,6.24,7.84];
	
			X = sort(X);
	
			%     a)
			Xmin = X(1);
			fprintf('Xmin = %.3f\n', Xmin);
			Xmax = X(end);
			fprintf('Xmax = %.3f\n', Xmax);
	
			%     b)
			R = Xmax - Xmin; 
			fprintf('R = %.3f\n', R);
	
			%     v)
			MX = expectation(X);
			fprintf('MX= %.3f\n', MX);
			DX = variance(X);
			fprintf('DX= %.3f\n', DX);
	
			%     g)
			m = group(X); 
			fprintf('m=%d\n', m);
			delta=R/m;
			fprintf('delta = %.3f\n', delta);
			J = Xmin : delta : Xmax;
			n = length(J);
			for i = 1:(n - 2)
				fprintf('[%.3f; %.3f)\n', J(i), J(i + 1));
			end
			fprintf('[%.3f; %.3f]\n', J(n - 1), J(n));
	
			sigma = sqrt(DX);
			Xn = (MX - R) : delta / 50 :(MX + R); 
	
		%     d)
			f(X, Xn, MX, sigma);
	
		%     e)
			F(X, Xn, MX, sigma);
	
	end
	
	function MX = expectation(X)
			MX = mean(X);
	end
	
	function DX = variance(X)
			DX = var(X);
	end
	
	function m = group(X)
			m = floor(log2(length(X))) + 2;
	end
	
	function F(X, Xn, MX, sigma)
			figure;
			Ycdf = 1/2 * (1 + erf((Xn - MX) / sqrt(2) * sigma));
			ecdf (X);
			hold on;
			plot(Xn, Ycdf, 'r'); 
			hold off ;
	end
	
	function f(X, Xn, MX, sigma)
			Ypdf = normpdf(Xn, MX, sigma);
			histogram (X, 'normalization' , 'pdf' ) ; 
			hold on;
			plt = plot(Xn, Ypdf, 'r');
			plt.LineWidth = 1;
	end
	
	\end{lstlisting}
	
	\vspace{0.5cm}
	\subsection{Результат выполнения программы}
	
	\begin{equation*}
	M_{\max} = 9.12
	\end{equation*}
	
	\begin{equation*}
	M_{\min} = 3.68
	\end{equation*}
	
	\begin{equation*}
	R = 5.44
	\end{equation*}
	
	\begin{equation*}
	\mu = 6.4596
	\end{equation*}
	
	\begin{equation*}
	S^2 = 1.1013
	\end{equation*}
	
	\begin{table}[H]
		\centering
		\begin{tabular}{|c|c|}
			\hline
			$[3.68;4.36)$ & 3 \\
			\hline
			$[4.36;5.04)$ & 8 \\
			\hline
			$[5.04;5.72)$ & 20 \\
			\hline
			$[5.72;6.40)$ & 22 \\
			\hline
			$[6.40;7.08)$ & 30 \\
			\hline
			$[7.08;7.76)$ & 25 \\
			\hline
			$[7.76;8.44)$ & 9 \\
			\hline
			$[8.44;9.12]$ & 3 \\
			\hline
		\end{tabular}
		\caption{Результаты расчетов для выборки}
	\end{table}

	\newpage


	\begin{figure}[H]
		\centering
		\includegraphics[scale=0.52]{img/Gistogramma.png}
		\caption{Гистограмма}
	\end{figure}

	\newpage
	
	\begin{figure}[H]
		\centering
		\includegraphics[scale=0.52]{img/EmpericalFunc.png}
		\caption{Эмпирическая функция распределения}
	\end{figure}
	
	
		
	
\end{document}