\documentclass[12pt, a4paper]{report}
\usepackage[utf8x]{inputenc}
\usepackage[T2A]{fontenc}
\usepackage[english,russian]{babel}

\usepackage{graphicx}
\usepackage{listings}
\usepackage{color}

\usepackage{amsfonts}
\usepackage{amsmath}
\usepackage{pgfplots}
\usepackage{url}
\usepackage{flowchart}
\usepackage{float}
\usepackage{tikz}
\usepackage{multirow}
\usepackage{graphicx}
\usepackage[
bookmarks=true, colorlinks=true, unicode=true,
urlcolor=black,linkcolor=black, anchorcolor=black,
citecolor=black, menucolor=black, filecolor=black,
]{hyperref}

\DeclareGraphicsExtensions{.pdf,.png,.jpg,.svg}
\usetikzlibrary{shapes, arrows}

\usepackage{pgfplotstable}

\renewcommand\contentsname{Содержание}

\usepackage{geometry}
\geometry{top=15mm}
\geometry{right=15mm}
\geometry{left=15mm}
\geometry{bottom=20mm}
\geometry{ignorefoot}


\lstset{ %
language=Lisp,                 % выбор языка для подсветки (здесь это С)
basicstyle=\small\sffamily, % размер и начертание шрифта для подсветки кода
numbers=left,               % где поставить нумерацию строк (слева\справа)
numberstyle=\tiny,           % размер шрифта для номеров строк
stepnumber=1,                   % размер шага между двумя номерами строк
numbersep=-5pt,                % как далеко отстоят номера строк от         подсвечиваемого кода
backgroundcolor=\color{white}, % цвет фона подсветки - используем         \usepackage{color}
showspaces=false,            % показывать или нет пробелы специальными     отступами
showstringspaces=false,      % показывать или нет пробелы в строках
showtabs=false,             % показывать или нет табуляцию в строках
frame=single,              % рисовать рамку вокруг кода
tabsize=2,                 % размер табуляции по умолчанию равен 2 пробелам
captionpos=t,              % позиция заголовка вверху [t] или внизу [b] 
breaklines=true,           % автоматически переносить строки (да\нет)
breakatwhitespace=false, % переносить строки только если есть пробел
escapeinside={\%*}{*)},   % если нужно добавить комментарии в коде
keywordstyle=\color{blue}\ttfamily,
stringstyle=\color{red}\ttfamily,
commentstyle=\color{green}\ttfamily,
morecomment=[l][\color{magenta}]{\#},
columns=fullflexible }

\usepackage{titlesec, blindtext, color}
\setcounter{secnumdepth}{-1}
\titleformat{\chapter}[hang]{\LARGE\bfseries}{}{0pt}{\LARGE\bfseries}
\titleformat{\section}{\bfseries\Large}{}{-1cm}{\centering}
\titleformat{\subsection}{\bfseries\normalsize}{}{-1cm}{\centering}



\begin{document}

  \begin{titlepage}
  	
  	\begin{table}[H]
  		\centering
  		\footnotesize
  		\begin{tabular}{cc}
  			\multirow{8}{*}{\includegraphics[scale=0.35]{img/bmstu.jpg}}
  			& \\
  			& \\
  			& \textbf{Министерство науки и высшего образования Российской Федерации} \\
  			& \textbf{Федеральное государственное бюджетное образовательное учреждение} \\
  			& \textbf{высшего образования} \\
  			& \textbf{<<Московский государственный технический} \\
  			& \textbf{университет имени Н.Э. Баумана>>} \\
  			& \textbf{(МГТУ им. Н.Э. Баумана)} \\
  			& \textbf{} \\
  		\end{tabular}
  	\end{table}
  	
  	\vspace{-2.5cm}
  	
  	\begin{flushleft}
  		\rule[-1cm]{\textwidth}{3pt}
  		\rule{\textwidth}{1pt}
  	\end{flushleft}
  	
  	\begin{flushleft}
  		\small
  		ФАКУЛЬТЕТ
  		\underline{<<Информатика и системы управления>>\ \ \ \ \ \ \ 
  			\ \ \ \ \ \ \ \ \ \ \ \ \ \ \ \ \ \ \ \ \ \ \ \ \ \ \ \ \ \ \ 
  			\ \ \ \ \ \ \ \ \ \ \ \ \ \ \ } \\
  		КАФЕДРА
  		\underline{<<Программное обеспечение ЭВМ и
  			информационные технологии>>
  			\ \ \ \ \ \ \ \ \ \ \ \ \ \ \ \ \ \ \ \ }
  	\end{flushleft}
  	
  	\vspace{2cm}
  	
  	\begin{center}
  		\textbf{Домашняя работа № 1} \\
  		\vspace{0.5cm}
  		\textbf{Вариант 22}
  	\end{center}
  	
  	\vspace{4cm}
  	
  	\begin{flushleft}
  		\begin{tabular}{ll}
  			\textbf{Дисциплина} & Математическая статистика. \\
  			\textbf{Тема} & \\
  			\textbf{Студент} & Тимонин А. С. \\
  			\textbf{Группа} & ИУ7-62Б \\
  			\textbf{Оценка (баллы)} & \\
  			\textbf{Преподаватель} & Власов П.А. \\
  		\end{tabular}
  	\end{flushleft}
  	
  	\vspace{6cm}
  	
  	\begin{center}
  		Москва, 2020 г.
  	\end{center}
  	
  	
  \end{titlepage}
  
	
	\section{ЗАДАЧА 1}
	
	\hspace{1cm} Известно, что 80\% изготовленных заводом электроламп выдерживают гарантийный срок службы. Найти вероятность того, что в партии из 500 электроламп число выдержавших гарантийный срок службы находится в пределах от 380 до 420. Использовать неравенство Чебышева и интегральную теорему Муавраз-Лапласа.
	
	\subsection{Неравенство Чебышева}
	
	\hspace{0.6cm}Пусть
	
	\begin{enumerate}
		\item $X$ - случайная величина
		\item $\exists MX,\ \exists DX$
	\end{enumerate}

	Тогда
	
	\begin{equation*}
	\forall \varepsilon > 0, P\{ |X - MX| \geq \varepsilon \} \le \frac{DX}{\varepsilon^2}
	\end{equation*}
	
	
	
	\textbf{Решение}
	
	\begin{itemize}
		\item $k_n$  - число успехов серии по схеме Бернулли;
		\item Вероятность успеха $p = 0.8$;
		\item $M[X] = np = 500 * 0.8 = 400$;
		\item $D[X] = npq = 500 * 0.8 * 0.2 = 80$;
	\end{itemize}


	\begin{multline*}
	P \{ 380 \le k_n \le 420 \} = P \{ -20 \le X - M[X] \le 20 \} = P \{ -20 \le X - 400 \le 20 \} = \\
    P \{ |X - 400| \le 20 \} \geq 1 - \frac{D[X]}{\varepsilon^2} \geq 1 - \frac{80}{400} \geq 1 - 0.2 \geq 0.8
	\end{multline*}
	
	\vspace{0.5cm}
	\textbf{Ответ:}
	$P \{ 380 \le k_n \le 420 \}$ = 0.8
	
	
	\vspace{0.5cm}\subsection{Центральная теорема Муавра-Лапласа}
	
	\hspace{0.6cm}Пусть
	
	\begin{enumerate}
		\item Проводится большое число испытаний по схеме Бернулли с вероятностью успеха p;
		\item k - число успехов этой серии.
	\end{enumerate}

	Тогда
	
	\begin{equation*}
	P\{ k_1 \le k \le k_2 \} = \Phi(x_2) - \Phi(x_1)
%	\forall \varepsilon > 0, P\{ |X - MX| \geq \varepsilon \} \le \frac{DX}{\varepsilon^2}
	\end{equation*}
	
	где $x_i$ = $\frac{k_i - np}{\sqrt{npq}}$, $i = \overline{1;2}$ , $q = 1 - p$
	
	
	\vspace{0.2cm}
	\textbf{Решение}
	
	\begin{enumerate}
		\item n = 500;
		\item Вероятность успеха $p = 0.8$;
		\item Вероятность неудачи $q = 0.2$;
		\item $k_1 = 380$, $k_2 = 420$;
	\end{enumerate}
	
	\vspace{0.5cm}
	
	\begin{equation*}
	x_1 = \frac{380 - 500 * 0.8}{\sqrt{500 * 0.8 * 0.2}} \approx -2.2361
	\end{equation*}
	
	\vspace{0.2cm}
	\begin{equation*}
	x_2 = \frac{420 - 500 * 0.8}{\sqrt{500 * 0.8 * 0.2}} \approx 2.2361
	\end{equation*}
	
	\vspace{0.7cm}
	
	\begin{equation*}
	P\{ 380 \le k \le 420 \} = \Phi(2.2361) + \Phi(2.2361) = 2*\Phi(2.2361) =  2 * 0.4873 = 0.9746
	\end{equation*}
	
	\vspace{0.5cm}
	\textbf{Ответ:}
	$P\{ 380 \le k \le 420 \} \approx 0.9746$
	
	
	
	
	
	
	\newpage
	
	\section{ЗАДАЧА 2}
	
	\hspace{1cm}
	С использованием метода моментов для случайной выборки $X = (X_1, ..., X_n)$ из генеральной совокупности X найти точечные оценки указанных параметров заданного закона распеделения.
	
	
	\vspace{0.5cm}
	\textbf{Закон распределения}
	
	\begin{equation*} 
	f_X(x) = \frac{1}{4^\theta \Gamma(\theta)}  x^{\theta - 1} e^{\frac{-x}{4}}, \;  x > 0
	\end{equation*}
	
	
	\vspace{0.2cm}
	\textbf{Решение}
	
	
	\vspace{0.2cm}
	1. Закон распределения является гаммой-функцией 
	\begin{equation*} 
	G_{k, \theta}(x) = x^{k-1} \frac{e^{\frac{-x}{\theta}}}{\Gamma(k)\theta^k} \Rightarrow k = \theta,\; \theta = 4
	\end{equation*} 
	
	\begin{equation*} 
	G_{\theta, 4}(x) = x^{\theta-1} \frac{e^{\frac{-x}{4}}}{\Gamma(\theta)4^\theta} 
	\end{equation*} 
	
	Тогда математическое ожидание и десперсию можно найти по формулам 
	
	\begin{equation*} 
	 M[G_{k ,\theta}] = k \theta
	\end{equation*} 
	
	\begin{equation*} 
	D[G_{k ,\theta}] = k \theta^2
	\end{equation*} 
	
	Найдем математическое ожидание и дисперсию
	
	\begin{equation*} 
	M[G_{\theta ,4}] =  4 * \theta
	\end{equation*}  
	
	\begin{equation*} 
	D[G_{\theta ,4}] = 16 * \theta
	\end{equation*}  
	
	
	\vspace{0.2cm}
	2. Приравняем теоретические моменты к их выборочным аналогам
	
	\begin{equation*} 
	4 \theta = \overline{X} \Rightarrow \theta = \frac{\overline{X}}{4}
	\end{equation*}  
	
	\begin{equation*} 
	\overline{X} = \frac{1}{n} \sum_{i=1}^n X_i
	\end{equation*} 
	
	\vspace{0.5cm}
	\textbf{Ответ:} $\theta = \frac{\overline{X}}{4}$
	

	
	\newpage
	
	\section{ЗАДАЧА 3}
	
	\hspace{1cm} С использованием метода максимального правдоподобия для случайно выборки $\vec{X} = (X_1, ..., X_n)$ из генеральной совокупности X найти точечные оценки параметров заданного закона распределения. Вычислить выборочные значения найденных оценок для выборки $\vec{x}_5 = (x_1, ..., x_5)$.
	
	\vspace{0.5cm}
	\textbf{Закон распределения}
	
	\begin{equation*}
	f_X(x) = \frac{\theta^5}{4!}  x^4  e^{-\theta x}, x > 0
	\end{equation*}
	
	\textbf{Выборка $\vec x_5$}
	
	\begin{equation*}
	(7, 4, 11, 5, 3)
	\end{equation*}
	
	\vspace{0.2cm}
	\textbf{Решение}
	
	\begin{equation*}
	L(\vec{X}, \vec{\theta}) = f(X_1, \vec{\theta}) \cdot ... \cdot f(X_n, \vec{\theta}) = \frac{\theta^{5n}}{4! \cdot n} \cdot (X_1 \cdot ... \cdot X_n)^4 \cdot e^{(-X_1 \cdot \theta)  + ... +(-X_n \cdot \theta) } )
	\end{equation*}
	
	\begin{equation*}
	\ln{L} = 5n \ln{\theta} - \ln{(4! \cdot n)} +  4 \cdot \ln{(X_1 \cdot ... \cdot X_n)} - \theta \cdot (X_1 + ... + X_n)
	\end{equation*}
	
	Воспользуемся необходимым условием экстремума
	
	\begin{equation*}
	\frac{\partial \ln L}{\partial \theta} = \frac{5n}{\theta} - (X_1 + ... + X_n) = 0
	\end{equation*}
	
	\begin{equation*}
	(X_1 + ... + X_n) = \frac{5n}{\theta}
	\end{equation*}
	
	\begin{equation*}
	\theta = \frac{5n}{(X_1 + ... + X_n)}
	\end{equation*}
	
	
	Покажем, что для найденных значений выполняются достаточные условия экстремума
	
	\begin{equation*}
	\frac{\partial^2 \ln L}{\partial^2 \theta} = -\frac{5n}{\theta^2} = -(X_1 + ... + X_n) < 0
	\end{equation*}
	
	Подставим выборку $\vec x_5$
	
	\begin{equation*}
	\theta = \frac{5 \cdot 5}{7 + 4 + 11 + 5 + 3} = 0.8(33)
	\end{equation*}
	
	\vspace{0.5cm}
	\textbf{Ответ:} 0.8(33)
	
	
	
	
	
	
	
	
	\newpage
	
	\section{ЗАДАЧА 4}
	
	\hspace{0.4cm} После $n = 8$ измерений давления в баке с горючим получены следующие результаты:
	
	\begin{equation*}
	3.25,\; 2.82,\; 3.07,\; 3.12,\; 2.93,\; 2.87,\; 3.09,\; 3.17.
	\end{equation*}
	
	Считая ошибки измерений подчиненными номральному закону, построить 90\%-ные доверительные интервалы для математического ожидания и среднего квадратичного отклонения давления в баке.
	
	
	\vspace{0.2cm}
	\textbf{Решение}
	
	Пусть X - случайная величина принимаюащая значения равные давлению в баке с горючим, так как
	
	\begin{equation*}
	Так как X \sim \mathbb{N}(m, \sigma^2) \text{ по условию } \Rightarrow
	\end{equation*}
	
	\begin{equation*}
	T(\vec{X}, m) = \frac{m-\overline{X}}{S(\vec{X})} \sqrt{n} \sim St(n-1)
	\end{equation*}
	
	\begin{equation*}
	P \{ t_{\alpha_1} < T(\vec{X}, m) < t_{1-\alpha_2} \} = \gamma
	\end{equation*}
	
	где $t_{\alpha_1}$, $t_{1 - \alpha_2}$ - квантили соотв. уровней распределения Стьюдента с $n-1 = 7$
	
	\begin{equation*}
	P \{  \overline{X} - \frac{S(\vec{X}) t_{\frac{1 + \gamma}{2}}}{\sqrt{n}} < m < \overline{X} + \frac{S(\vec{X}) t_{\frac{1 + \gamma}{2}}}{\sqrt{n}}   \} = \gamma
	\end{equation*}
	
	\begin{equation*}
	\underline{m} (\vec{X}) = \overline{X} - \frac{S(\vec{X}) \frac{t_1+ \gamma}{2}}{\sqrt{n}}
	\end{equation*}
	
	\begin{equation*}
	\overline{m} (\vec{X}) = \overline{X} + \frac{S(\vec{X}) \frac{t_1+ \gamma}{2}}{\sqrt{n}}
	\end{equation*}
	
	\vspace{0.5cm}
	
	\begin{equation*}
	\textbf{а)}\; \overline{X} = 3.04
	\end{equation*}
	
	\begin{equation*}
	\textbf{б)}\; S^2(\overline{X}) = \frac{1}{n-1} \sum_{i=1}^{n}(X_i - \overline{X})^2 \approx 0.0229 \Rightarrow S(\vec{X}) = 0.1513
	\end{equation*}
	
	\begin{equation*}
	\textbf{в)}\; \frac{1+\gamma}{2} = 0.95
	\end{equation*}
	
	\begin{equation*}
	\textbf{г)}\; t_{0.95} = 1.895
	\end{equation*}
	
	\begin{equation*}
	\textbf{д)}\; \frac{S(\vec{X}) t_{\frac{1 + \alpha}{2}}{\sqrt{n}}} = \frac{0.1513  \cdot  1.895}{2.83} \approx 0.1013
	\end{equation*}
	
	\begin{equation*}
	\textbf{г)}\; \underline{m} (\vec{X}) = 3.04 - 0.1013 = 2.9387
	\end{equation*}
	
	\begin{equation*}
	\overline{m} (\vec{X}) = 3.1413 \Rightarrow m \in (2.9387; 3.1413)
	\end{equation*}
	
	 \vspace{0.5cm} Так как $X \sim \mathbb{N}(m, \sigma^2)$ по условию $\Rightarrow$
	 
	 \begin{equation*}
	 \Rightarrow T(\vec{X}, \sigma^2) =  \frac{S^2(\vec{X})}{\sigma^2} (n-1) \sim \chi^2(n-1) 
	 \end{equation*}
	 
	 \begin{equation*}
	 P \{   h_{\alpha_1} < T(\vec{X}, \sigma^2) < h_{1 - \alpha_2}  \} = \gamma
	 \end{equation*}
	 
	 где $h_{\alpha_1}$, $h_{1 - \alpha_2}$ n квантили соответствующих уровней распределения $\chi^2$ c $n-1=7$
	 
	 \begin{equation*}
	 P \{    \frac{  (n-1) S^2(\vec{X})  }{h_{  \frac{1+\gamma}{2}}} < \sigma^2 < \frac{   (n-1)S^2(\vec{X})    }{ h_{\frac{1-\gamma}{2}}      }    \} = \gamma
	 \end{equation*}
	 
	 \begin{equation*}
	 \underline{\sigma^2}(\vec{X}) = \frac{(n-1)S^2(\vec{X})}{h_{\frac{1+\gamma}{2}}}
	 \end{equation*}
	 
	 \begin{equation*}
	 \overline{\sigma^2}(\vec{X}) = \frac{(n-1)S^2(\vec{X})}{h_{\frac{1-\gamma}{2}}}
	 \end{equation*}
	  
	  \begin{equation*}
	  \textbf{а)}\; S^2(\vec{X}) = 0.0229
	  \end{equation*}
	  
	  \begin{equation*}
	  \textbf{б)}\; \frac{1+\gamma}{2} = 0.95
	  \end{equation*}
	  
	  \begin{equation*}
	  \frac{1-\gamma}{2} = 0.05
	  \end{equation*}
	  
	  \begin{equation*}
	  \textbf{в)}\; h_{0.95} = 14.07
	  \end{equation*}
	  
	  \begin{equation*}
	  h_{0.05} = 2.17
	  \end{equation*}
	  
	  \begin{equation*}
	  \textbf{г)}\; \underline{\sigma}^2(\vec{X}) = \frac{  (n-1) S^2(\vec{X})}{  h_{\frac{1+\gamma}{2}}  } = \frac{7 \cdot 0.0229}{14.07} = 0.0114
	  \end{equation*}
	  
	  \begin{equation*}
	  \overline{\sigma}^2(\vec{X}) = \frac{  (n-1) S^2(\vec{X})}{  h_{\frac{1-\gamma}{2}}  } = \frac{7 \cdot 0.0229}{2.17} = 0.074
	  \end{equation*}
	
	\vspace{0.5cm}
	\textbf{Ответ:} $\sigma^2 \in (0.0114; 0.074)$
	
	
	

\end{document}