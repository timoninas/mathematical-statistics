\documentclass[12pt, a4paper]{report}
\usepackage[utf8x]{inputenc}
\usepackage[T2A]{fontenc}
\usepackage[english,russian]{babel}

\usepackage{graphicx}
\usepackage{listings}
\usepackage{color}

\usepackage{amsfonts}
\usepackage{amsmath}
\usepackage{pgfplots}
\usepackage{url}
\usepackage{flowchart}
\usepackage{float}
\usepackage{tikz}
\usepackage{multirow}
\usepackage{graphicx}
\usepackage[
bookmarks=true, colorlinks=true, unicode=true,
urlcolor=black,linkcolor=black, anchorcolor=black,
citecolor=black, menucolor=black, filecolor=black,
]{hyperref}

\DeclareGraphicsExtensions{.pdf,.png,.jpg,.svg}
\usetikzlibrary{shapes, arrows}

\usepackage{pgfplotstable}

\renewcommand\contentsname{Содержание}

\usepackage{geometry}
\geometry{top=15mm}
\geometry{right=15mm}
\geometry{left=15mm}
\geometry{bottom=20mm}
\geometry{ignorefoot}


\lstset{ %
language=Lisp,                 % выбор языка для подсветки (здесь это С)
basicstyle=\small\sffamily, % размер и начертание шрифта для подсветки кода
numbers=left,               % где поставить нумерацию строк (слева\справа)
numberstyle=\tiny,           % размер шрифта для номеров строк
stepnumber=1,                   % размер шага между двумя номерами строк
numbersep=-5pt,                % как далеко отстоят номера строк от         подсвечиваемого кода
backgroundcolor=\color{white}, % цвет фона подсветки - используем         \usepackage{color}
showspaces=false,            % показывать или нет пробелы специальными     отступами
showstringspaces=false,      % показывать или нет пробелы в строках
showtabs=false,             % показывать или нет табуляцию в строках
frame=single,              % рисовать рамку вокруг кода
tabsize=2,                 % размер табуляции по умолчанию равен 2 пробелам
captionpos=t,              % позиция заголовка вверху [t] или внизу [b] 
breaklines=true,           % автоматически переносить строки (да\нет)
breakatwhitespace=false, % переносить строки только если есть пробел
escapeinside={\%*}{*)},   % если нужно добавить комментарии в коде
keywordstyle=\color{blue}\ttfamily,
stringstyle=\color{red}\ttfamily,
commentstyle=\color{green}\ttfamily,
morecomment=[l][\color{magenta}]{\#},
columns=fullflexible }

\usepackage{titlesec, blindtext, color}
\setcounter{secnumdepth}{-1}
\titleformat{\chapter}[hang]{\LARGE\bfseries}{}{0pt}{\LARGE\bfseries}
\titleformat{\section}{\bfseries\Large}{}{-1cm}{\centering}
\titleformat{\subsection}{\bfseries\normalsize}{}{-1cm}{\centering}



\begin{document}

  \begin{titlepage}
  	
  	\begin{table}[H]
  		\centering
  		\footnotesize
  		\begin{tabular}{cc}
  			\multirow{8}{*}{\includegraphics[scale=0.35]{img/bmstu.jpg}}
  			& \\
  			& \\
  			& \textbf{Министерство науки и высшего образования Российской Федерации} \\
  			& \textbf{Федеральное государственное бюджетное образовательное учреждение} \\
  			& \textbf{высшего образования} \\
  			& \textbf{<<Московский государственный технический} \\
  			& \textbf{университет имени Н.Э. Баумана>>} \\
  			& \textbf{(МГТУ им. Н.Э. Баумана)} \\
  			& \textbf{} \\
  		\end{tabular}
  	\end{table}
  	
  	\vspace{-2.5cm}
  	
  	\begin{flushleft}
  		\rule[-1cm]{\textwidth}{3pt}
  		\rule{\textwidth}{1pt}
  	\end{flushleft}
  	
  	\begin{flushleft}
  		\small
  		ФАКУЛЬТЕТ
  		\underline{<<Информатика и системы управления>>\ \ \ \ \ \ \ 
  			\ \ \ \ \ \ \ \ \ \ \ \ \ \ \ \ \ \ \ \ \ \ \ \ \ \ \ \ \ \ \ 
  			\ \ \ \ \ \ \ \ \ \ \ \ \ \ \ } \\
  		КАФЕДРА
  		\underline{<<Программное обеспечение ЭВМ и
  			информационные технологии>>
  			\ \ \ \ \ \ \ \ \ \ \ \ \ \ \ \ \ \ \ \ }
  	\end{flushleft}
  	
  	\vspace{2cm}
  	
  	\begin{center}
  		\textbf{Домашняя работа № 2} \\
  		\vspace{0.5cm}
  		\textbf{Вариант 22}
  	\end{center}
  	
  	\vspace{4cm}
  	
  	\begin{flushleft}
  		\begin{tabular}{ll}
  			\textbf{Дисциплина} & Математическая статистика. \\
  			\textbf{Тема} & \\
  			\textbf{Студент} & Тимонин А. С. \\
  			\textbf{Группа} & ИУ7-62Б \\
  			\textbf{Оценка (баллы)} & \\
  			\textbf{Преподаватель} & Власов П.А. \\
  		\end{tabular}
  	\end{flushleft}
  	
  	\vspace{6cm}
  	
  	\begin{center}
  		Москва, 2020 г.
  	\end{center}
  	
  	
  \end{titlepage}
  
	
	\section{ЗАДАЧА}
	
	\hspace{1cm} Известно, что 80\% изготовленных заводом электроламп выдерживают гарантийный срок службы. Найти вероятность того, что в партии из 500 электроламп число выдержавших гарантийный срок службы находится в пределах от 380 до 420. Использовать неравенство Чебышева и интегральную теорему Муавраз-Лапласа.
	
	\subsection{Неравенство Чебышева}
	
	\begin{equation*}
	\forall \varepsilon > 0, P\{ |X - MX| \geq \varepsilon \} \le \frac{DX}{\varepsilon^2}
	\end{equation*}
	
	
	
	\textbf{Решение}
	
	\vspace{0.5cm}
	\textbf{Ответ:}
	$P \{ 380 \le k_n \le 420 \}$ = 0.8
	
	\newpage
	

\end{document}